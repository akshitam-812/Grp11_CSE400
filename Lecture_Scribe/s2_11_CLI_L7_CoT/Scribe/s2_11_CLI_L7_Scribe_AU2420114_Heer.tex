\documentclass[12pt]{article}

\usepackage{amsmath, amssymb}
\usepackage{geometry}
\usepackage{setspace}

\geometry{margin=1in}
\onehalfspacing

\title{\textbf{CSE400 – Fundamentals of Probability in Computing}\\
Lecture 7: Expectation, CDFs, PDFs, and Problem Solving}
\author{Source: Lecture slides by Dhaval Patel, PhD}
\date{January 27, 2025}

\begin{document}

\maketitle

\section{The Cumulative Distribution Function (CDF)}

\subsection{Definition of the CDF}

Let $X$ be a random variable.  
The \textbf{Cumulative Distribution Function (CDF)} of $X$ is defined as
\[
F_X(x) = \Pr(X \le x), \quad -\infty < x < \infty.
\]

For every real number $x$, the function $F_X(x)$ assigns the probability that the random variable $X$ takes a value less than or equal to $x$.  
Thus, the CDF is a mapping from the real line to probabilities.

The lecture explicitly notes that \textbf{most of the information about the random experiment described by the random variable $X$ is determined by the behavior of $F_X(x)$}.

\subsection{Properties of the CDF}

\subsubsection*{Property 1: Boundedness}

\[
0 \le F_X(x) \le 1.
\]

\textbf{Reasoning:}  
Since $F_X(x) = \Pr(X \le x)$ is a probability, by the axioms of probability it must lie between $0$ and $1$ inclusive.

\subsubsection*{Property 2: Limits at Infinity}

\[
F_X(-\infty) = 0, \quad F_X(\infty) = 1.
\]

\textbf{Reasoning:}
\begin{itemize}
    \item As $x \to -\infty$, the event $\{X \le x\}$ becomes impossible, so its probability approaches $0$.
    \item As $x \to \infty$, the event $\{X \le x\}$ becomes certain, so its probability approaches $1$.
\end{itemize}

\subsubsection*{Property 3: Monotonicity}

For $x_1 < x_2$,
\[
F_X(x_1) \le F_X(x_2).
\]

\textbf{Reasoning:}  
If $x_1 < x_2$, then the event $\{X \le x_1\}$ is a subset of the event $\{X \le x_2\}$.  
Since probabilities are monotone with respect to set inclusion, the inequality follows.  
Hence, a valid CDF must be non-decreasing.

\subsubsection*{Property 4: Probability Over an Interval}

For $x_1 < x_2$,
\[
\Pr(x_1 < X \le x_2) = F_X(x_2) - F_X(x_1).
\]

\textbf{Reasoning:}  
The event $\{x_1 < X \le x_2\}$ can be written as the difference between the events $\{X \le x_2\}$ and $\{X \le x_1\}$.  
Subtracting their probabilities yields the stated result.

\subsection{Example: Validity of Candidate CDFs}

\subsubsection*{Candidate 1}

\[
F_X(x) = \frac{1}{2} + \frac{1}{\pi}\tan^{-1}(x).
\]

\begin{itemize}
    \item As $x \to -\infty$, $\tan^{-1}(x) \to -\frac{\pi}{2}$, hence $F_X(x) \to 0$.
    \item As $x \to \infty$, $\tan^{-1}(x) \to \frac{\pi}{2}$, hence $F_X(x) \to 1$.
    \item The function is monotone increasing.
\end{itemize}

Therefore, this function satisfies all CDF properties and is valid.

\subsubsection*{Candidate 2}

\[
F_X(x) = \left(1 - e^{-x}\right)u(x),
\]
where $u(x)$ is the unit step function.

\begin{itemize}
    \item For $x < 0$, $u(x) = 0$, so $F_X(x) = 0$.
    \item For $x \ge 0$, $F_X(x) = 1 - e^{-x}$, which increases from $0$ to $1$.
    \item The function is non-decreasing and bounded between $0$ and $1$.
\end{itemize}

Hence, this is a valid CDF.

\subsubsection*{Candidate 3}

\[
F_X(x) = e^{-x^2}.
\]

\begin{itemize}
    \item As $x \to -\infty$, $e^{-x^2} \to 0$.
    \item As $x \to \infty$, $e^{-x^2} \to 0$, not $1$.
\end{itemize}

This violates the property $F_X(\infty) = 1$, so it is not a valid CDF.

\section{The Probability Density Function (PDF)}

\subsection{Definition of the PDF}

For a continuous random variable, the \textbf{Probability Density Function (PDF)} of $X$ at point $x$ is defined as
\[
f_X(x) = \lim_{\epsilon \to 0} \frac{\Pr(x \le X \le x + \epsilon)}{\epsilon}.
\]

This definition considers the probability that $X$ lies in a small interval of width $\epsilon$ around $x$, normalized by the interval length.

\subsection{Relationship Between PDF and CDF}

For a continuous range,
\[
\Pr(x \le X \le x + \epsilon) = F_X(x + \epsilon) - F_X(x).
\]

Substituting into the PDF definition,
\[
f_X(x) = \lim_{\epsilon \to 0} \frac{F_X(x + \epsilon) - F_X(x)}{\epsilon}.
\]

By the definition of the derivative,
\[
f_X(x) = \frac{dF_X(x)}{dx}.
\]

\subsection{Logical Dependency Between PDF and CDF}

\begin{itemize}
    \item The PDF of a random variable is the derivative of its CDF.
    \item The CDF of a random variable can be expressed as the integral of its PDF.
\end{itemize}

\section{Expectation of Random Variables}

\subsection{Definition of Expectation}

Expectation is a numerical summary associated with a random variable.  
It is defined with respect to the distribution of the random variable and is used to compute average or mean values.

\subsection{Expectation of a Function of a Random Variable}

The expectation of a function of $X$ is computed by applying the same probabilistic weighting defined for $X$ itself.

\subsection{Linear Operations with Expectation}

Expectation is linear under addition and scalar multiplication.  
This allows expectations of sums or scaled random variables to be computed systematically.

\section{Moments and Central Moments of Random Variables}

\subsection{n-th Moments}

The $n$-th moments are higher-order expectations that characterize the distribution of a random variable.

\subsection{Central Moments}

Central moments are defined relative to the mean and include:
\begin{itemize}
    \item Variance
    \item Skewness
    \item Kurtosis
\end{itemize}

These quantities provide measures of spread, asymmetry, and tail behavior.

\section{Summary}

Lecture 7 establishes:
\begin{itemize}
    \item The formal definition and properties of the CDF
    \item Methods to verify whether a function is a valid CDF
    \item The definition of the PDF and its derivation from the CDF
    \item The derivative--integral relationship between PDF and CDF
    \item The concept of expectation and higher-order moments
\end{itemize}

\end{document}
