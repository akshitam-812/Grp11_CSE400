\documentclass[12pt]{article}
\usepackage{amsmath}

\begin{document}

CSE400 -- Fundamentals of Probability in Computing

Lecture 7: Expectation, CDFs, PDFs and Problem Solving

\section*{1. Cumulative Distribution Function (CDF)}

\subsection*{1.1 Definition of CDF}

Let $X$ be a random variable.

The Cumulative Distribution Function (CDF) of the random variable $X$ is defined as:

\[
F_X(x) = \Pr(X \le x), \quad -\infty < x < \infty
\]

This definition states that for any real number $x$, the CDF gives the probability that the random variable $X$ takes a value less than or equal to $x$.

\subsection*{1.2 Importance of the CDF}

Most of the information about the random experiment described by the random variable $X$ is determined by the behavior of its CDF $F_X(x)$.

\subsection*{1.3 Properties of the CDF}

The CDF satisfies the following properties:

Property 1: Boundedness

\[
0 \le F_X(x) \le 1
\]

This follows from the fact that probabilities are always between 0 and 1.

Property 2: Limits at Infinity

\[
F_X(-\infty) = 0, \quad F_X(\infty) = 1
\]

This means:

As $x \to -\infty$, the probability that $X \le x$ approaches 0.

As $x \to \infty$, the probability that $X \le x$ approaches 1.

Property 3: Monotonicity

For $x_1 < x_2$,

\[
F_X(x_1) \le F_X(x_2)
\]

This shows that the CDF is a non-decreasing (monotonic) function.

Property 4: Interval Probability Using CDF

For $x_1 < x_2$,

\[
\Pr(x_1 < X \le x_2) = F_X(x_2) - F_X(x_1)
\]

This property allows probabilities over intervals to be computed using differences of CDF values.

\subsection*{1.4 Example \#1: Validity of a CDF}

Problem:

Determine which of the following functions are valid CDFs.

(a)

\[
F_X(x) = \frac{1}{2} + \frac{1}{\pi} \tan^{-1}(x)
\]

As $x \to -\infty$, $\tan^{-1}(x) \to -\frac{\pi}{2}$, hence $F_X(x) \to 0$.

As $x \to \infty$, $\tan^{-1}(x) \to \frac{\pi}{2}$, hence $F_X(x) \to 1$.

Function is non-decreasing

Conclusion: Valid CDF

(b)

\[
F_X(x) = [1 - e^{-x}]u(x)
\]

For $x < 0$, $u(x) = 0 \Rightarrow F_X(x) = 0$

For $x \ge 0$, function increases monotonically

Limit as $x \to \infty$ is 1

Conclusion: Valid CDF

(c)

\[
F_X(x) = e^{-x^2}
\]

As $x \to \infty$, $F_X(x) \to 0$

Does not satisfy $F_X(\infty) = 1$

Conclusion: Not a valid CDF

(d)

\[
F_X(x) = x^2 u(x)
\]

As $x \to \infty$, function diverges

Violates boundedness property

Conclusion: Not a valid CDF

\subsection*{1.5 Example \#2: Probability Computation Using CDF}

Given:

\[
F_X(x) = (1 - e^{-x})u(x)
\]

(a)

\[
\Pr(X > 5)
\]

\[
\Pr(X > 5) = 1 - \Pr(X \le 5) = 1 - F_X(5)
\]

\[
= 1 - (1 - e^{-5}) = e^{-5}
\]

(b)

\[
\Pr(X < 5)
\]

\[
\Pr(X < 5) = F_X(5) = 1 - e^{-5}
\]

(c)

\[
\Pr(3 < X < 7)
\]

\[
= F_X(7) - F_X(3)
\]

\[
= (1 - e^{-7}) - (1 - e^{-3}) = e^{-3} - e^{-7}
\]

(d)

\[
\Pr(X > 5 \mid X < 7)
\]

\[
= \frac{\Pr(5 < X < 7)}{\Pr(X < 7)}
\]

\[
= \frac{F_X(7) - F_X(5)}{F_X(7)}
\]

\section*{2. Probability Density Function (PDF)}

\subsection*{2.1 Definition of PDF}

For a continuous random variable $X$, the Probability Density Function (PDF) is defined as:

\[
f_X(x) =
\lim_{\epsilon \to 0}
\frac{\Pr(x \le X \le x + \epsilon)}{\epsilon}
\]

\subsection*{2.2 Relationship Between PDF and CDF}

Recall for a continuous random variable:

\[
\Pr(x \le X \le x + \epsilon)
=
F_X(x + \epsilon) - F_X(x)
\]

Substituting into the PDF definition:

\[
f_X(x)
=
\lim_{\epsilon \to 0}
\frac{F_X(x + \epsilon) - F_X(x)}{\epsilon}
\]

This is the definition of the derivative.

\subsection*{2.3 Final Relationship}

\[
f_X(x) = \frac{dF_X(x)}{dx}
\]

Thus:

The PDF is the derivative of the CDF

The CDF is the integral of the PDF

\[
F_X(x) = \int_{-\infty}^{x} f_X(t)\, dt
\]

End of Lecture 7 Scribe (as covered in provided PPT)

\end{document}