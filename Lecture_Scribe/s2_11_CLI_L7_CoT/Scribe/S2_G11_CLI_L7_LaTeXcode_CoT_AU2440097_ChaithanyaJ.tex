
\documentclass[12pt]{article}
\usepackage[a4paper, margin=1in]{geometry}
\usepackage{amsmath, amssymb}
\usepackage{enumitem}
\usepackage{setspace}

\setstretch{1.2}

\begin{document}

\begin{center}
    \Large \textbf{CSE400 – Fundamentals of Probability in Computing} \\[0.3cm]
    \large \textbf{Lecture 7: Expectation, CDFs, PDFs and Problem Solving}
\end{center}

\vspace{0.5cm}

\noindent
\textbf{Instructor:}
\begin{itemize}[leftmargin=1.5cm]
    \item Dhaval Patel, PhD
    \item Associate Professor
    \item Major Advisor – Computer Science and Engineering (CSE)
    \item Machine Intelligence, Computing and xG Networks (MICxN) Research Lab
    \item SEAS – Ahmedabad University, Ahmedabad, Gujarat, India
\end{itemize}

\noindent
\textbf{Date:} January 27, 2025

\vspace{0.5cm}

%--------------------------------------------------

\section*{Part I: Lecture Outline}

The lecture is organized into the following topics:

\begin{itemize}
    \item The Cumulative Density Function (CDF)
    \begin{itemize}
        \item Definition
        \item Properties
        \item Example
    \end{itemize}

    \item The Probability Density Function (PDF)
    \begin{itemize}
        \item Definition
        \item Properties
        \item Example
    \end{itemize}

    \item Expectation of Random Variables (RVs)
    \begin{itemize}
        \item Definition and Example
        \item Expectation of a Function of RV
        \item Linear Operation with Expectation
    \end{itemize}

    \item nth Moments and Central Moments of Random Variables
    \begin{itemize}
        \item Variance
        \item Skewness
        \item Kurtosis
    \end{itemize}
\end{itemize}

%--------------------------------------------------

\section*{Part II: The Cumulative Density Function (CDF)}

\subsection*{Topic Heading}
\begin{itemize}
    \item The Cumulative Density Function (CDF)
\end{itemize}

\subsection*{Subsections}
\begin{itemize}
    \item Definition
    \item Properties
    \item Example
\end{itemize}

%--------------------------------------------------

\section*{Part III: The Probability Density Function (PDF)}

\subsection*{Topic Heading}
\begin{itemize}
    \item The Probability Density Function (PDF)
\end{itemize}

\subsection*{Subsections}
\begin{itemize}
    \item Definition
    \item Properties
    \item Example
\end{itemize}

%--------------------------------------------------

\section*{Part IV: Expectation of Random Variables (RVs)}

\subsection*{Topic Heading}
\begin{itemize}
    \item Expectation of RVs
\end{itemize}

\subsection*{Subsections}
\begin{itemize}
    \item Definition and Example
    \item Expectation of a Function of RV
    \item Linear Operation with Expectation
\end{itemize}

%--------------------------------------------------

\section*{Part V: nth Moments and Central Moments of Random Variables}

\subsection*{Topic Heading}
\begin{itemize}
    \item nth Moments and Central Moments of RVs
\end{itemize}

\subsection*{Components}
\begin{itemize}
    \item Variance
    \item Skewness
    \item Kurtosis
\end{itemize}

%--------------------------------------------------

\section*{Part VI: Structural Notes on the Source Material}

\begin{itemize}
    \item The lecture slides consist primarily of topic headings and outlines.
    \item No explicit mathematical derivations, proofs, or formulas are presented in the available textual content.
    \item Repetition of the outline across multiple slides reinforces the thematic structure of the lecture.
\end{itemize}

\vspace{0.5cm}

\begin{center}
    \textbf{End of Lecture Scribe}
\end{center}

\end{document}
```
