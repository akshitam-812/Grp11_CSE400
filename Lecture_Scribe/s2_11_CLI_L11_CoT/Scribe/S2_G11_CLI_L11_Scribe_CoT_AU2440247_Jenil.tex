\documentclass[11pt]{article}

\usepackage[a4paper,margin=1in]{geometry}
\usepackage{amsmath,amssymb,amsthm}
\usepackage{mathtools}
\usepackage{hyperref}
\usepackage{enumitem}

\title{\textbf{CSE400 -- Fundamentals of Probability in Computing}\\
\large Lecture 11: Transformation of Random Variables}
\author{Dhaval Patel, PhD (Associate Professor)\\
SEAS -- Ahmedabad University, Ahmedabad, Gujarat, India}
\date{February 10, 2026}

\begin{document}
\maketitle

% Source: Uploaded lecture PDF. :contentReference[oaicite:0]{index=0}

\section*{Overview / Outline (as in lecture)}
\begin{enumerate}[label=\arabic*.]
    \item \textbf{Transformation of Random Variables}\\
    Learning of transformation techniques for random variables.
    \item \textbf{Function of Two Random Variables}\\
    Joint transformations and derived distributions.
    \item \textbf{Illustrative Example}\\
    Detailed derivation for the case: $Z = X + Y$.
\end{enumerate}

\section{Definitions \& Notation}
\begin{itemize}[leftmargin=1.2em]
    \item Random variable (RV): $X$, $Y$, $Z$.
    \item Transformation (single RV): \quad $Y = g(X)$.
    \item CDF of a RV $X$: \quad $F_X(x) = \Pr(X \le x)$.
    \item PDF of a continuous RV $X$: \quad $f_X(x)$, with the relationship
    \[
    f_X(x) = \frac{d}{dx}F_X(x)\quad \text{(when differentiable)}.
    \]
    \item Inverse mapping (when it exists): \quad $x = g^{-1}(y)$.
\end{itemize}

\section{Assumptions / Constraints (as used in the lecture derivations)}
\begin{itemize}[leftmargin=1.2em]
    \item The PDF of the original RV (e.g., $f_X(x)$) is assumed to be known \emph{a priori}.
    \item For the single-variable transformation method shown:
    \begin{itemize}
        \item $g(\cdot)$ is \textbf{monotonic} (lecture separates cases: $+ve$ / increasing vs $-ve$ / decreasing).
        \item $g^{-1}(\cdot)$ exists on the relevant range.
        \item Differentiation steps assume the needed derivatives exist.
    \end{itemize}
    \item If a detail/condition is not explicitly provided beyond the above, it is \textbf{Not specified in the provided context.}
\end{itemize}

\section{Main Results / Theorems (with conditions)}
\subsection{Single RV transformation: $Y=g(X)$ (monotonic $g$)}
\textbf{Step S1 (CDF method).}

\paragraph{Case 1: $g$ is increasing (\textit{``+ve'' monotone})}
\begin{align}
F_Y(y) 
&= \Pr(Y \le y) \\
&= \Pr(g(X)\le y) \\
&= \Pr\bigl(X \le g^{-1}(y)\bigr) \\
&= F_X\bigl(g^{-1}(y)\bigr).
\end{align}

\paragraph{Case 2: $g$ is decreasing (\textit{``-ve'' monotone})}
\begin{align}
F_Y(y)
&= \Pr(Y \le y) \\
&= \Pr(g(X)\le y) \\
&= \Pr\bigl(X \ge g^{-1}(y)\bigr) \\
&= 1 - F_X\bigl(g^{-1}(y)\bigr).
\end{align}

\textbf{Step S2 (Differentiate to obtain PDF).}

\paragraph{Increasing case}
Differentiate $F_Y(y)=F_X(g^{-1}(y))$ w.r.t.\ $y$:
\begin{align}
f_Y(y) 
&= \frac{d}{dy}F_Y(y) \\
&= \frac{d}{dy}\Bigl[F_X\bigl(g^{-1}(y)\bigr)\Bigr] \\
&= f_X\bigl(g^{-1}(y)\bigr)\cdot \frac{d}{dy}\bigl[g^{-1}(y)\bigr].
\end{align}
Let $x=g^{-1}(y)$. Then $\dfrac{dx}{dy}=\dfrac{d}{dy}g^{-1}(y)$, so
\[
f_Y(y)= f_X(x)\left|\frac{dx}{dy}\right| \quad \text{with } x=g^{-1}(y).
\]

\paragraph{Decreasing case}
Starting from $F_Y(y)=1-F_X(g^{-1}(y))$:
\begin{align}
f_Y(y)
&= \frac{d}{dy}F_Y(y) \\
&= -\, f_X\bigl(g^{-1}(y)\bigr)\cdot \frac{d}{dy}\bigl[g^{-1}(y)\bigr].
\end{align}
Taking absolute value (as emphasized in the lecture’s boxed formula), the unified form is
\[
f_Y(y)= f_X(x)\left|\frac{dx}{dy}\right|\quad \text{with } x=g^{-1}(y).
\]

\textbf{Step S3 (Change the limits / support).}\\
The valid range of $y$ is obtained by mapping the support of $x$ through $y=g(x)$ (lecture notes: ``change the limits for $Y$'').

\subsection{Equivalent boxed formula shown in the lecture}
Using $\left|\dfrac{dx}{dy}\right| = \dfrac{1}{\left|\dfrac{dy}{dx}\right|}$ (when derivatives exist),
\[
\boxed{
f_Y(y)= \frac{f_X(x)}{\left|\dfrac{dy}{dx}\right|}\biggr|_{x=g^{-1}(y)}
}
\]
This is the boxed relationship displayed in the lecture.

\section{Proofs / Derivations (step-by-step, as in lecture)}
\subsection{Derivation structure for $Y=g(X)$}
The lecture’s derivation proceeds in the following explicit sequence:
\begin{enumerate}[label=\textbf{S\arabic*:}, leftmargin=2.2em]
    \item Compute $F_Y(y)=\Pr(Y\le y)$ and rewrite the event in terms of $X$ using monotonicity and the inverse $g^{-1}$.
    \item Differentiate $F_Y(y)$ w.r.t.\ $y$ to obtain $f_Y(y)$, applying the chain rule.
    \item Determine the correct support/range for $y$ by transforming the original $x$-limits through $y=g(x)$.
\end{enumerate}

\subsection{Two-RV function setup used for the example $Z=X+Y$}
The lecture sets up the CDF of $Z$ via a region in the $(x,y)$-plane:
\begin{align}
F_Z(z)
&= \Pr(Z \le z) \\
&= \Pr(X+Y \le z).
\end{align}
Using the joint PDF $f_{X,Y}(x,y)$, the probability is written as a double integral over the region
\[
\{(x,y): x+y\le z\}.
\]
One explicit ``vertical strip'' form written in the lecture is:
\[
F_Z(z)=\int_{x=-\infty}^{\infty}\int_{y=-\infty}^{z-x} f_{X,Y}(x,y)\,dy\,dx.
\]
(An equivalent ``horizontal strip'' order is implied by the lecture’s annotation about vertical/horizontal swapping, but any further simplification is \textbf{Not specified in the provided context}.)

Then, to obtain the PDF:
\[
f_Z(z)=\frac{d}{dz}F_Z(z).
\]
Any closed-form $f_Z(z)$ beyond this setup depends on additional assumptions/distributions, which are listed as sub-questions in the lecture (see Worked Examples).

\section{Worked Examples (step-by-step)}
\subsection{Example 1: $X\sim \text{Uniform}(-1,1)$, $Y=g(X)=\sin\left(\frac{\pi X}{2}\right)$}
\textbf{Given (as written in the lecture):}
\[
f_X(x)=
\begin{cases}
\frac{1}{2}, & -1<x<1,\\
0, & \text{otherwise}.
\end{cases}
\qquad
Y=\sin\left(\frac{\pi X}{2}\right).
\]

\textbf{Step 1: Invert the transformation.}\\
From $y=\sin\left(\frac{\pi x}{2}\right)$,
\[
x = \frac{2}{\pi}\sin^{-1}(y).
\]

\textbf{Step 2: Compute the derivative factor.}
\[
\frac{dx}{dy}
= \frac{2}{\pi}\cdot \frac{1}{\sqrt{1-y^2}}.
\]

\textbf{Step 3: Apply the transformation formula.}
Using $f_Y(y)=f_X(x)\left|\frac{dx}{dy}\right|$ with $x=\frac{2}{\pi}\sin^{-1}(y)$:
\begin{align}
f_Y(y)
&= \frac{1}{2}\cdot \left|\frac{2}{\pi}\cdot \frac{1}{\sqrt{1-y^2}}\right| \\
&= \frac{1}{\pi\sqrt{1-y^2}}.
\end{align}

\textbf{Step 4: Determine the support (change limits).}\\
Lecture maps endpoints:
\[
x=-1 \Rightarrow y=\sin\left(-\frac{\pi}{2}\right)=-1,
\qquad
x=1 \Rightarrow y=\sin\left(\frac{\pi}{2}\right)=1.
\]
Hence,
\[
f_Y(y)=
\begin{cases}
\dfrac{1}{\pi\sqrt{1-y^2}}, & -1<y<1,\\[6pt]
0, & \text{otherwise}.
\end{cases}
\]

\subsection{Example 2 (illustrative setup): $Z=X+Y$}
The lecture explicitly states the following items to find/prove (no full solutions shown beyond the CDF-region setup):
\begin{enumerate}[label=(\roman*), leftmargin=2.2em]
    \item Find the PDF of $Z$, $f_Z(z)$.
    \item Find $f_Z(z)$ if $X$ and $Y$ are independent.
    \item Let $X\sim N(0,1)$ and $Y\sim N(0,1)$. Prove that $Z\sim N(0,2)$.
    \item If $X$ and $Y$ are exponential distributed RVs with parameter $\lambda$, find $f_Z(z)$.
\end{enumerate}

\textbf{Derivation shown (CDF setup):}
\begin{align}
F_Z(z)
&= \Pr(Z\le z) \\
&= \Pr(X+Y\le z) \\
&= \iint_{x+y\le z} f_{X,Y}(x,y)\,dx\,dy \\
&= \int_{-\infty}^{\infty}\int_{-\infty}^{z-x} f_{X,Y}(x,y)\,dy\,dx.
\end{align}
Further evaluation for cases (ii)--(iv) is \textbf{Not specified in the provided context} beyond listing these tasks.

\section*{Summary of Key Takeaways (only from context)}
\begin{itemize}[leftmargin=1.2em]
    \item To find the distribution of a transformed RV $Y=g(X)$ (with monotonic $g$), the lecture uses the \textbf{CDF method} followed by differentiation.
    \item For monotonic transformations, the resulting PDF uses the absolute derivative factor:
    \[
    f_Y(y)= f_X\bigl(g^{-1}(y)\bigr)\left|\frac{d}{dy}g^{-1}(y)\right|
    \quad \text{equivalently}\quad
    f_Y(y)=\frac{f_X(x)}{\left|\frac{dy}{dx}\right|}\Big|_{x=g^{-1}(y)}.
    \]
    \item The support of the new RV must be obtained by \textbf{changing limits} (mapping $x$-range through $y=g(x)$).
    \item For $Z=X+Y$, the lecture sets up $F_Z(z)=\Pr(X+Y\le z)$ as a \textbf{double integral over the half-plane} $x+y\le z$ using $f_{X,Y}(x,y)$, and then uses $f_Z(z)=\dfrac{d}{dz}F_Z(z)$.
\end{itemize}

\end{document}
