\documentclass[12pt]{article}
\usepackage{amsmath, amssymb}
\usepackage{geometry}
\geometry{margin=1in}
\usepackage{setspace}
\setstretch{1.2}

\begin{document}

\section*{CSE400 -- Fundamentals of Probability in Computing}

\subsection*{Lecture 11: Transformation of Random Variables}

\textbf{Instructor:} Dhaval Patel \\
\textbf{Course:} CSE400 -- Fundamentals of Probability in Computing \\
\textbf{Affiliation:} SEAS-Ahmedabad University, Ahmedabad, Gujarat, India \\
\textbf{Date:} February 10, 2026

\hrule
\vspace{1em}

\section{Transformation of Random Variables}

\subsection{Objective}

\begin{itemize}
    \item To learn transformation techniques for random variables.
    \item To determine the distribution of a transformed random variable.
    \item To extend the idea to functions involving two random variables.
    \item To derive the distribution in the illustrative case:
    \begin{itemize}
        \item $Z = X + Y$
    \end{itemize}
\end{itemize}

\section{Transformation of a Random Variable}

Let:
\begin{itemize}
    \item $X$ be a random variable.
    \item A new random variable $Z$ be defined as a function of $X$, i.e.,
    \[
        Z = g(X)
    \]
\end{itemize}

The objective is:
\begin{itemize}
    \item To determine the distribution of $Z$ from the known distribution of $X$.
\end{itemize}

\subsection{General Method (CDF-Based Approach)}

To find the distribution of $Z$:
\begin{enumerate}
    \item Define:
    \[
        F_Z(z) = P(Z \leq z)
    \]
    \item Since $Z = g(X)$, substitute:
    \[
        F_Z(z) = P(g(X) \leq z)
    \]
    \item Express the event $g(X) \leq z$ in terms of $X$.
    \item Use the known distribution of $X$ to compute the probability.
    \item Once $F_Z(z)$ is obtained, differentiate (if continuous) to obtain the PDF:
    \[
        f_Z(z) = \frac{d}{dz} F_Z(z)
    \]
\end{enumerate}

\section{Function of Two Random Variables}

Let:
\begin{itemize}
    \item $X$ and $Y$ be two random variables.
    \item Define a new random variable:
    \[
        Z = g(X, Y)
    \]
\end{itemize}

The objective is:
\begin{itemize}
    \item To determine the distribution of $Z$ from the joint distribution of $X$ and $Y$.
\end{itemize}

\subsection{Joint Distribution Framework}

If:
\begin{itemize}
    \item $f_{X,Y}(x,y)$ is the joint PDF of $X$ and $Y$,
\end{itemize}

Then:
\begin{itemize}
    \item Probabilities involving $Z$ are computed using the joint distribution.
\end{itemize}

\subsection{CDF of $Z$}

To determine the distribution of $Z$:
\begin{enumerate}
    \item Define:
    \[
        F_Z(z) = P(Z \leq z)
    \]
    \item Substitute:
    \[
        F_Z(z) = P(g(X,Y) \leq z)
    \]
    \item Express the event $g(X,Y) \leq z$ as a region in the $(x,y)$-plane.
    \item Compute:
    \[
        F_Z(z) = \iint_{g(x,y) \leq z} f_{X,Y}(x,y)\, dx\, dy
    \]
    \item If continuous, differentiate:
    \[
        f_Z(z) = \frac{d}{dz} F_Z(z)
    \]
\end{enumerate}

\section{Illustrative Example: $Z = X + Y$}

Let:
\[
    Z = X + Y
\]

Objective:
\begin{itemize}
    \item To derive the distribution of $Z$.
\end{itemize}

\subsection{CDF of $Z$}

By definition:
\[
    F_Z(z) = P(Z \leq z)
\]

Substitute:
\[
    F_Z(z) = P(X + Y \leq z)
\]

This corresponds to:
\begin{itemize}
    \item The set of all $(x,y)$ such that:
    \[
        x + y \leq z
    \]
\end{itemize}

Thus:
\[
    F_Z(z) = \iint_{x+y \leq z} f_{X,Y}(x,y)\, dx\, dy
\]

\subsection{Derivation of the PDF}

To obtain the PDF:
\[
    f_Z(z) = \frac{d}{dz} F_Z(z)
\]

Using the integral expression:
\[
    F_Z(z) = \iint_{x+y \leq z} f_{X,Y}(x,y)\, dx\, dy
\]

Differentiating with respect to $z$ yields:
\[
    f_Z(z) = \int_{-\infty}^{\infty} f_{X,Y}(x, z - x)\, dx
\]

This provides the distribution of $Z = X + Y$.

\section{Logical Flow of the Lecture}

\begin{itemize}
    \item Start with transformation of a single random variable.
    \item Extend to functions involving two random variables.
    \item Use the cumulative distribution function as the primary method.
    \item Convert probability statements into integrals over appropriate regions.
    \item Differentiate to obtain the PDF.
    \item Apply the method to the specific case:
    \[
        Z = X + Y
    \]
\end{itemize}

\section{Key Formulas}

\subsection*{Single Random Variable}

\[
    F_Z(z) = P(g(X) \leq z)
\]

\[
    f_Z(z) = \frac{d}{dz} F_Z(z)
\]

\subsection*{Two Random Variables}

\[
    F_Z(z) = \iint_{g(x,y) \leq z} f_{X,Y}(x,y)\, dx\, dy
\]

\subsection*{Special Case: $Z = X + Y$}

\[
    F_Z(z) = P(X + Y \leq z)
\]

\[
    f_Z(z) = \int_{-\infty}^{\infty} f_{X,Y}(x, z-x)\, dx
\]

\vspace{1em}
\noindent\textbf{End of Lecture 11 Scribe}

\end{document}
