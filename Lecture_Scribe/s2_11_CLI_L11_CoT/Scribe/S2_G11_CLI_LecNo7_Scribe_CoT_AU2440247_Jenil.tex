\documentclass[12pt]{article}

\usepackage{amsmath, amssymb}
\usepackage{geometry}
\usepackage{setspace}
\usepackage{hyperref}

\geometry{margin=1in}
\onehalfspacing

\title{\textbf{CSE400: Fundamentals of Probability in Computing}\\
Lecture 7: Expectation, CDFs, PDFs and Problem Solving}
\author{Dhaval Patel, PhD\\
Associate Professor, CSE\\
Ahmedabad University}
\date{January 27, 2025}

\begin{document}
\maketitle

\tableofcontents
\newpage

%------------------------------------------------------------
\section{Cumulative Distribution Function (CDF)}

\subsection{Definition}

Let $X$ be a random variable.  
The \textbf{Cumulative Distribution Function (CDF)} of $X$ is defined as:
\[
F_X(x) = \Pr(X \leq x), \quad -\infty < x < \infty
\]

The behavior of $F_X(x)$ determines most of the information about the random experiment described by the random variable $X$.

\subsection{Properties of the CDF}

The CDF $F_X(x)$ satisfies the following properties:

\begin{enumerate}
    \item \textbf{Bounds:}
    \[
    0 \leq F_X(x) \leq 1
    \]

    \item \textbf{Limits at Infinity:}
    \[
    F_X(-\infty) = 0, \quad F_X(\infty) = 1
    \]

    \item \textbf{Monotonicity:}  
    For $x_1 < x_2$,
    \[
    F_X(x_1) \leq F_X(x_2)
    \]

    \item \textbf{Probability Over an Interval:}  
    For $x_1 < x_2$,
    \[
    \Pr(x_1 < X \leq x_2) = F_X(x_2) - F_X(x_1)
    \]
\end{enumerate}

These properties are necessary for a function to be a valid CDF.

%------------------------------------------------------------
\subsection{Example 1: Validity of CDFs}

Determine whether each of the following functions is a valid CDF.

\subsubsection*{(a)}
\[
F_X(x) = \frac{1}{2} + \frac{1}{\pi} \tan^{-1}(x)
\]

\textbf{Verification:}
\begin{itemize}
    \item $\tan^{-1}(x)$ is monotone increasing.
    \item $\tan^{-1}(-\infty) = -\frac{\pi}{2}$ and $\tan^{-1}(\infty) = \frac{\pi}{2}$.
\end{itemize}

Thus,
\[
F_X(-\infty) = 0, \quad F_X(\infty) = 1
\]

Hence, this is a valid CDF.

\subsubsection*{(b)}
\[
F_X(x) = \left[1 - e^{-x}\right]u(x)
\]

where $u(x)$ is the unit step function.

\textbf{Verification:}
\begin{itemize}
    \item $F_X(x)=0$ for $x<0$.
    \item $F_X(x)$ is non-decreasing.
    \item $\lim_{x\to\infty} F_X(x) = 1$.
\end{itemize}

Hence, this is a valid CDF.

\subsubsection*{(c)}
\[
F_X(x) = e^{-x^2}
\]

This function does not satisfy the limit condition:
\[
\lim_{x\to\infty} F_X(x) \neq 1
\]

Hence, it is \textbf{not} a valid CDF.

%------------------------------------------------------------
\subsection{Example 2: Probability Computation Using a CDF}

Suppose a random variable has CDF:
\[
F_X(x) = (1 - e^{-x})u(x)
\]

Compute the following probabilities.

\subsubsection*{(a) $\Pr(X > 5)$}

\[
\Pr(X > 5) = 1 - \Pr(X \leq 5) = 1 - F_X(5)
\]

\[
= 1 - (1 - e^{-5}) = e^{-5}
\]

\subsubsection*{(b) $\Pr(X < 5)$}

\[
\Pr(X < 5) = F_X(5) = 1 - e^{-5}
\]

\subsubsection*{(c) $\Pr(3 < X < 7)$}

\[
\Pr(3 < X < 7) = F_X(7) - F_X(3)
\]

\[
= (1 - e^{-7}) - (1 - e^{-3}) = e^{-3} - e^{-7}
\]

\subsubsection*{(d) $\Pr(X > 5 \mid X < 7)$}

Using conditional probability:
\[
\Pr(A \mid B) = \frac{\Pr(A \cap B)}{\Pr(B)}
\]

\[
\Pr(X > 5 \mid X < 7) =
\frac{\Pr(5 < X < 7)}{\Pr(X < 7)}
\]

\[
= \frac{F_X(7) - F_X(5)}{F_X(7)}
= \frac{(1 - e^{-7}) - (1 - e^{-5})}{1 - e^{-7}}
\]

%------------------------------------------------------------
\section{Probability Density Function (PDF)}

\subsection{Definition}

For a continuous random variable $X$, the \textbf{Probability Density Function (PDF)} is defined as:
\[
f_X(x) = \lim_{\epsilon \to 0}
\frac{\Pr(x \leq X < x + \epsilon)}{\epsilon}
\]

\subsection{PDF--CDF Relationship}

For a continuous random variable:
\[
\Pr(x \leq X < x + \epsilon) = F_X(x+\epsilon) - F_X(x)
\]

Substituting into the definition:
\[
f_X(x) = \lim_{\epsilon \to 0}
\frac{F_X(x+\epsilon) - F_X(x)}{\epsilon}
\]

Thus,
\[
\boxed{f_X(x) = \frac{d}{dx}F_X(x)}
\]

\subsection{Inverse Relationship}

Conversely, the CDF can be obtained from the PDF by integration:
\[
F_X(x) = \int_{-\infty}^{x} f_X(t)\,dt
\]

Hence:
\begin{itemize}
    \item The PDF is the derivative of the CDF.
    \item The CDF is the integral of the PDF.
\end{itemize}

%------------------------------------------------------------
\section{Expectation of Random Variables}

The lecture outline introduces the following topics:
\begin{itemize}
    \item Expectation of random variables
    \item Expectation of a function of a random variable
    \item Linear operation with expectation
    \item $n^{\text{th}}$ moments and central moments (variance, skewness, kurtosis)
\end{itemize}

\textbf{Note:}  
The provided lecture slides do not include definitions, formulas, derivations, or examples for these topics.  
Therefore, no additional content is reconstructed here.

\end{document}
