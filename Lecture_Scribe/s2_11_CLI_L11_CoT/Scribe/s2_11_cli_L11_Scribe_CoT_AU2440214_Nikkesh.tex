\documentclass[12pt]{article}

\usepackage[a4paper, margin=1in]{geometry}
\usepackage{amsmath, amssymb}
\usepackage{setspace}
\usepackage{titlesec}
\usepackage{enumitem}

\setstretch{1.15}
\setlength{\parskip}{6pt}
\setlength{\parindent}{0pt}

\begin{document}

%------------------------------------------------
% Title Section
%------------------------------------------------

\begin{center}
    {\Large \textbf{CSE400 – Fundamentals of Probability in Computing}} \\[8pt]
    {\large \textbf{Lecture 11: Transformation of Random Variables}} \\[12pt]
    Dhaval Patel, PhD \\
    Associate Professor \\
    SEAS – Ahmedabad University \\
    February 10, 2026
\end{center}

\vspace{10pt}
\hrule

%------------------------------------------------
% Outline
%------------------------------------------------

\section*{Outline}

\begin{enumerate}
    \item \textbf{Transformation of Random Variables}  
    Learning of transformation techniques for random variables.
    
    \item \textbf{Function of Two Random Variables}  
    Joint transformations and derived distributions.
    
    \item \textbf{Illustrative Example}  
    Detailed derivation for the case: $Z = X + Y$
\end{enumerate}

\vspace{10pt}
\hrule
\vspace{15pt}

%------------------------------------------------
% Section 1
%------------------------------------------------

\section*{1. Transformation of Random Variables}

\textbf{Assumption}

The PDF of random variable $X$, denoted $f_X(x)$, is known \textit{a priori}.

Objective: To find the PDF of a new random variable

\[
Y = g(X)
\]

%---------------------------
\subsection*{Case 1: Monotonic Transformation}

Assume $g(x)$ is monotonic (either strictly increasing or strictly decreasing).

\textbf{Step S1: Find the CDF of $Y$}

\[
F_Y(y) = \Pr(Y \le y)
\]

Since $Y = g(X)$,

\[
F_Y(y) = \Pr(g(X) \le y)
\]

If $g$ is monotonic increasing,

\[
g(X) \le y \iff X \le g^{-1}(y)
\]

\[
F_Y(y) = F_X(g^{-1}(y))
\]

\textbf{Step S2: Differentiate to obtain PDF}

\[
f_Y(y) = \frac{d}{dy}F_Y(y)
\]

\[
= f_X(g^{-1}(y)) \cdot \frac{d}{dy}g^{-1}(y)
\]

\[
= f_X(x)\left|\frac{dx}{dy}\right|_{x=g^{-1}(y)}
\]

%---------------------------
\subsection*{Case 2: Monotonic Decreasing Function}

\[
g(X) \le y \iff X \ge g^{-1}(y)
\]

\[
F_Y(y) = 1 - F_X(g^{-1}(y))
\]

\[
f_Y(y) = f_X(x)\left|\frac{dx}{dy}\right|_{x=g^{-1}(y)}
\]

%---------------------------
\subsection*{Final Result (Monotonic Case)}

\[
\boxed{
f_Y(y) =
\frac{f_X(x)}
{\left| \frac{dy}{dx} \right|}
\Bigg|_{x = g^{-1}(y)}
}
\]

\vspace{15pt}
\hrule
\vspace{15pt}

%------------------------------------------------
% Section 2
%------------------------------------------------

\section*{2. Example: Transformation}

Given

\[
X \sim \text{Uniform}(-1,1)
\]

\[
f_X(x)=
\begin{cases}
\dfrac{1}{2}, & -1<x<1 \\
0, & \text{otherwise}
\end{cases}
\]

Transformation:

\[
Y = \sin\left(\frac{\pi X}{2}\right)
\]

\subsection*{Step 1: Inverse Transformation}

\[
X = \frac{2}{\pi}\sin^{-1}(Y)
\]

\subsection*{Step 2: Derivative}

\[
\frac{dx}{dy}=
\frac{2}{\pi}
\frac{1}{\sqrt{1-y^2}}
\]

\subsection*{Step 3: Apply Formula}

\[
f_Y(y)=
\frac{1}{2}
\cdot
\frac{2}{\pi}
\cdot
\frac{1}{\sqrt{1-y^2}}
=
\frac{1}{\pi\sqrt{1-y^2}}
\]
\subsection*{Step 4: Limits}
\[
f_Y(y)=
\begin{cases}
\dfrac{1}{\pi\sqrt{1-y^2}}, & -1<y<1 \\
0, & \text{otherwise}
\end{cases}
\]

\vspace{15pt}
\hrule
\vspace{15pt}

%------------------------------------------------
% Section 3
%------------------------------------------------

\section*{3. Function of Two Random Variables}

Consider
\[
Z = X + Y
\]
We are required to find:

\begin{enumerate}
    \item PDF of $Z$, $f_Z(z)$
    \item $f_Z(z)$ if $X$ and $Y$ are independent
    \item If $X \sim N(0,1)$ and $Y \sim N(0,1)$, prove $Z \sim N(0,2)$
    \item If $X$ and $Y$ are exponential with parameter $\lambda$, find $f_Z(z)$
\end{enumerate}

\subsection*{Illustrative Example: Derivation for $Z = X + Y$}

\textbf{Step 1: CDF}

\[
F_Z(z)=\Pr(X+Y \le z)
\]

\subsection*{Step 2: Double Integral}

\[
F_Z(z)=
\iint_{x+y \le z}
f_{X,Y}(x,y)\,dx\,dy
\]

\subsection*{Step 3: Horizontal Strip}

\[
F_Z(z)=
\int_{-\infty}^{\infty}
\int_{-\infty}^{z-y}
f_{X,Y}(x,y)\,dx\,dy
\]

\subsection*{Step 4: Vertical Strip}

\[
F_Z(z)=
\int_{-\infty}^{\infty}
\int_{-\infty}^{z-x}
f_{X,Y}(x,y)\,dy\,dx
\]

\subsection*{Step 5: Differentiate}

\[
f_Z(z)=
\frac{d}{dz}F_Z(z)
\]

\vspace{20pt}

\begin{center}
\textit{This completes the lecture material exactly as presented in Lecture 11.}
\end{center}

\end{document}
