\documentclass[12pt]{article}

\usepackage{amsmath, amssymb}
\usepackage{geometry}
\usepackage{setspace}

\geometry{margin=1in}
\onehalfspacing

\title{\textbf{CSE400 – Fundamentals of Probability in Computing}\\
Lecture 11: Transformation of Random Variables}
\author{}
\date{}

\begin{document}

\maketitle

\section{Transformation of One Random Variable}

Let $X$ be a continuous random variable with known probability density function (PDF) $f_X(x)$.  
Let a new random variable $Y$ be defined as a function of $X$:
\[
Y = g(X).
\]

The objective is to derive the PDF of $Y$, denoted $f_Y(y)$, given the PDF of $X$.

\subsection{Step 1: Start from the Cumulative Distribution Function (CDF)}

By definition, the CDF of $Y$ is
\[
F_Y(y) = \Pr(Y \le y).
\]

Substituting $Y = g(X)$,
\[
F_Y(y) = \Pr(g(X) \le y).
\]

\subsection{Step 2: Use Monotonicity of the Transformation}

The lecture considers monotonic transformations $g(x)$, which may be:
\begin{itemize}
    \item Monotonically increasing, or
    \item Monotonically decreasing.
\end{itemize}

This assumption is essential because it allows inversion of the function $g(x)$.

\subsubsection*{Case 1: $g(x)$ is Monotonically Increasing}

If $g(x)$ is increasing, then
\[
g(X) \le y \iff X \le g^{-1}(y).
\]

Hence,
\[
F_Y(y) = \Pr(X \le g^{-1}(y)) = F_X(g^{-1}(y)).
\]

\subsubsection*{Case 2: $g(x)$ is Monotonically Decreasing}

If $g(x)$ is decreasing, then
\[
g(X) \le y \iff X \ge g^{-1}(y).
\]

Thus,
\[
F_Y(y) = \Pr(X \ge g^{-1}(y)) = 1 - F_X(g^{-1}(y)).
\]

\subsection{Step 3: Differentiate the CDF to Obtain the PDF}

The PDF of $Y$ is obtained by differentiating the CDF with respect to $y$:
\[
f_Y(y) = \frac{d}{dy}F_Y(y).
\]

For the increasing case,
\[
f_Y(y) = \frac{d}{dy}\left[F_X(g^{-1}(y))\right].
\]

Applying the chain rule,
\[
f_Y(y) = f_X(g^{-1}(y)) \cdot \frac{d}{dy}\left[g^{-1}(y)\right].
\]

\subsection{Step 4: Role of the Absolute Derivative}

Both increasing and decreasing cases can be combined using the absolute value of the derivative:
\[
\boxed{
f_Y(y) = \frac{f_X(x)}{\left|\dfrac{dy}{dx}\right|}
\quad \text{evaluated at } x = g^{-1}(y)
}
\]

The absolute value accounts for:
\begin{itemize}
    \item Positive slope (increasing transformation),
    \item Negative slope (decreasing transformation).
\end{itemize}

Thus, the magnitude of stretching or compression of the transformation directly affects the density.

\section{Function of Two Random Variables}

Let $X$ and $Y$ be two continuous random variables with joint PDF $f_{X,Y}(x,y)$.  
Define a new random variable
\[
Z = X + Y.
\]

The objective is to derive the PDF $f_Z(z)$.

\subsection{Step 1: Define the CDF of $Z$}

By definition,
\[
F_Z(z) = \Pr(Z \le z).
\]

Substituting $Z = X + Y$,
\[
F_Z(z) = \Pr(X + Y \le z).
\]

\subsection{Step 2: Geometric Interpretation}

The condition $X + Y \le z$ represents a region in the $(x,y)$-plane bounded by:
\begin{itemize}
    \item The line $x + y = z$,
    \item The support of the joint PDF $f_{X,Y}(x,y)$.
\end{itemize}

The probability is computed by integrating the joint PDF over this region.

\subsection{Step 3: Express the CDF as a Double Integral}

From the lecture setup,
\[
F_Z(z) =
\int_{-\infty}^{\infty}
\int_{-\infty}^{z-x}
f_{X,Y}(x,y)\,dy\,dx.
\]

\subsection{Step 4: Differentiate to Obtain the PDF}

Differentiating with respect to $z$,
\[
f_Z(z) = \frac{d}{dz}F_Z(z).
\]

Differentiation with respect to the upper limit yields
\[
\boxed{
f_Z(z) = \int_{-\infty}^{\infty} f_{X,Y}(x, z-x)\,dx
}.
\]

\section{Illustrative Examples}

\subsection{Example 1: Transformation of a Uniform Random Variable}

Let
\begin{itemize}
    \item $X \sim \text{Uniform}(-1,1)$,
    \item $Y = \sin\!\left(\dfrac{\pi X}{2}\right)$.
\end{itemize}

\subsubsection*{Step 1: PDF of $X$}

\[
f_X(x) =
\begin{cases}
\dfrac{1}{2}, & -1 < x < 1,\\
0, & \text{otherwise}.
\end{cases}
\]

\subsubsection*{Step 2: Transformation and Inverse}

\[
y = \sin\!\left(\dfrac{\pi x}{2}\right)
\quad \Rightarrow \quad
x = \dfrac{2}{\pi}\sin^{-1}(y).
\]

\subsubsection*{Step 3: Derivative}

\[
\frac{dx}{dy}
= \frac{2}{\pi}\cdot\frac{1}{\sqrt{1-y^2}}.
\]

\subsubsection*{Step 4: Apply the Transformation Formula}

\[
f_Y(y) = f_X(x)\left|\frac{dx}{dy}\right|
= \frac{1}{2}\cdot\frac{2}{\pi\sqrt{1-y^2}}
= \frac{1}{\pi\sqrt{1-y^2}}.
\]

\subsubsection*{Step 5: Support of $Y$}

Since $x \in (-1,1)$ implies $y \in (-1,1)$,
\[
f_Y(y) =
\begin{cases}
\dfrac{1}{\pi\sqrt{1-y^2}}, & -1 < y < 1,\\
0, & \text{otherwise}.
\end{cases}
\]

\subsection{Example 2: Derivation for $Z = X + Y$}

The lecture follows these steps:
\begin{enumerate}
    \item Define $Z = X + Y$,
    \item Write $F_Z(z) = \Pr(X + Y \le z)$,
    \item Identify the region under the line $x + y = z$,
    \item Integrate the joint PDF over this region,
    \item Differentiate with respect to $z$.
\end{enumerate}

This yields
\[
\boxed{
f_Z(z) = \int_{-\infty}^{\infty} f_{X,Y}(x, z-x)\,dx
}.
\]

\end{document}
