\documentclass[12pt]{article}
\usepackage{amsmath,amssymb}
\usepackage{geometry}
\geometry{a4paper, margin=1in}

\begin{document}

\title{CSE400: Project Kickoff and Undergraduate Research Programme\\
Lecture Scribe}
\author{}
\date{}
\maketitle

\section{Course Component Overview}

The course CSE400 includes a project component with a total weightage of thirty percent. This project component is structured around defined milestones, scheduled submissions, and evaluative checkpoints distributed across the semester. The project is executed in teams formed at the beginning of the course.

The project kickoff lecture introduces the structure, expectations, milestones, submission formats, assessment mechanisms, and institutional research context associated with the course.

\section{Project Team Formation}

Project team formation is the first mandatory activity in the course project lifecycle. The deadline for team formation is specified as seventeenth January 2026, Saturday, end of day. All subsequent project submissions and evaluations are conducted on a per-group basis.

\section{Project Execution Guidelines}

The project carries a total weightage of thirty percent in the overall course assessment. The execution of the project is organized around six major milestones, denoted as M1 through M6. There are a total of six submissions throughout the semester, with one submission required per group for each milestone.

The deliverables associated with the project include codes, reports, videos, and other artifacts as specified at different stages of the project execution. Team assessment is conducted twice, once before the mid-semester evaluation and once after the mid-semester evaluation. A project viva and final submission are conducted towards the end of the course.

\section{Major Project Milestones}

The project milestones are defined sequentially and are intended to guide the progression of the project from initial conceptualization to final analysis and submission.

\subsection{Milestone M1}

Milestone M1 is titled ``Kickstart''. This milestone includes team formation, area identification, background study, motivation, and problem formulation. The outcome of this milestone establishes the foundational direction of the project.

\subsection{Milestone M2}

Milestone M2 focuses on mathematical modeling. The selected problem must be mathematically modeled in any chosen domain. The modeling includes random variables, probability mass functions or probability density functions, cumulative distribution functions, multivariate random variables, and joint probability mass, density, or cumulative distribution functions.

\subsection{Milestone M3}

Milestone M3 involves coding. This includes simulation and computation related to the mathematical model developed in the previous milestone.

\subsection{Milestone M4}

Milestone M4 is dedicated to inference. In this milestone, a randomized algorithm is chosen, understood, and implemented in code.

\subsection{Milestone M5}

Milestone M5 applies randomized algorithms to the selected domain problem. This milestone includes the application of randomized algorithms and the generation of results. New inferences are drawn in comparison to deterministic algorithms.

\subsection{Milestone M6}

Milestone M6 consists of deriving bounds, performing analysis, and compiling the final deliverables for submission.

\section{Submission \#1: Concept Evolution Maps}

The first submission requires the creation of concept evolution maps. The recommended tools for preparing concept evolution maps include Miro and diagrams.net. These maps are intended to visually represent the evolution of concepts relevant to the project.

\section{Submission \#2: Scribe and Learning Reflection Logs}

Submission number two consists of scribe preparation and learning reflection logs. This submission is mandatory and must strictly adhere to the specified tools and formats.

Two types of scribes are identified: lecture scribes and project scribes. For lecture scribes, every lecture is assigned to two groups who are responsible for generating a scribe reflecting the lecture content. The scribe must include additional examples from foreign textbooks and must be a minimum of eight to ten pages in length.

The process and decision-making documentation includes decision logs that record why one option was chosen over another, constraints, alternatives considered, final decisions, evidence used, and trade-off matrices comparing cost, performance, and risk.

There are a total of six such submissions throughout the semester, conducted in a bi-weekly mode. Each submission focuses on answering specific questions related to the project work.

\section{Submission \#3: Multimodal Artifacts}

Submission number three involves the creation of multimodal artifacts, which may include video, audio, or visual content. The emphasis is placed on the content being delivered rather than editing skills.

One video is to be prepared per milestone, with a duration of approximately ten to fifteen minutes. The video should explain the work completed in the corresponding milestone and include coding simulations where applicable. Presentation tools such as PowerPoint or Google Slides may be used for narration, and any recording tool may be used.

The accepted formats include think-aloud videos, one-minute insight videos, and project demos.

\section{Introduction to the Undergraduate Research Programme}

The lecture introduces the Undergraduate Research Programme (UGRP) for the academic years 2026 to 2027. Reference is made to Dr. BJ Fogg, founder of the Behavior Design Lab at Stanford University, where research and innovation are directed.

\section{Rationale for UGRP}

The motivation for UGRP is linked to the concept of the T-shaped engineer, as discussed by Joe Tranquillo in the Journal of Engineering Education Transformations. The IBM concept of the T-shaped individual is presented, where the vertical bar represents depth in a single technical discipline and the horizontal bar represents the ability to apply knowledge across disciplines and work with others.

\section{Philosophy for UGRP}

The philosophy of UGRP is multidisciplinary, encompassing arts, science, and management. It emphasizes experiential learning and is research-driven. The programme follows a four-dimensional model consisting of Discover, Design, Develop, and Deliver.

The breadth areas include computer science and computer science engineering. Data science with applied artificial intelligence, modern computer system design involving hardware and software, and networks including IoT, IoBNT, and IoV are highlighted.

\section{Industry and Research Activities}

Current industry activities include the design and development of a five-G enabled intelligent transportation systems testbed in Gujarat. This includes AI-based ITS solutions, V2X communication between on-board units and roadside units without internet connectivity, SDK testing, interfacing and sensor data collection, in-car dashboard development, and multiple research directions such as deep learning-based crash prevention, deep reinforcement learning-based smart signaling, and coexistence frameworks.

\section{International Bilateral Collaborations}

The lecture lists multiple international collaborators from universities in the United Kingdom, Singapore, Malaysia, Finland, Japan, Qatar, and other institutions. These collaborations support the research ecosystem associated with the programme.

\section{MICxN Research Lab Alumni}

Alumni of the MICxN Research Lab are presented along with their current professional affiliations, including roles at Amazon, Google, Intel, American Express, and academic institutions.

\section{Research Publications and Outcomes}

The lecture highlights publication venues, including international conferences such as IEEE PIMRC, COMSNETS, NCC, and IEEE Transactions on Vehicular Technology. Multiple undergraduate research programme batches are associated with accepted papers, best paper awards, and best undergraduate presentation awards.

\section{Current Students and Research Areas}

Current students associated with the MICxN Research Lab are listed along with their research areas, which include advanced spectrum sensing, O-RAN, six-G integrated sensing and communication with UAVs, Wi-Fi sensing, wireless networks, Wi-Fi 8, and antenna design.

\section{Conclusion}

The lecture concludes with an open question and answer session and provides contact information for further communication.

\end{document}
