\documentclass[12pt]{article}

\usepackage[a4paper,margin=1in]{geometry}
\usepackage{setspace}
\usepackage{amsmath,amssymb}
\usepackage{enumitem}
\usepackage{hyperref}

\setstretch{1.15}

\title{\textbf{CSE400 -- Fundamentals of Probability in Computing}\\
Lecture 3: Introduction to Probability Theory}
\author{Instructor: Dhaval Patel, PhD}
\date{January 13, 2026}

\begin{document}

\maketitle

\hrule
\vspace{1em}

\section{Lecture Context and Position in the Course}

This lecture is part of \textbf{CSE400: Fundamentals of Probability in Computing} and is explicitly titled \textbf{``Lecture 3: Introduction to Probability Theory.''} The lecture is positioned after the initial course orientation content and before the introduction of advanced probabilistic modeling and algorithmic applications.

The primary role of this lecture is to introduce probability theory as the foundational mathematical framework that will be used throughout the remainder of the course.

\section{Purpose and Motivation of Learning Probability}

The lecture emphasizes why probability theory is necessary in computing and engineering disciplines.

\subsection*{Logical Structure}

\begin{enumerate}
    \item Computing systems frequently operate under \textbf{uncertainty}.
    \item This uncertainty arises due to incomplete information, randomness, noise, or variability.
    \item Probability theory provides a formal mathematical language to:
    \begin{itemize}
        \item Represent uncertainty,
        \item Quantify likelihoods,
        \item Support rational decision-making.
    \end{itemize}
\end{enumerate}

This motivation is presented prior to any formal definitions, establishing probability theory as a prerequisite analytical tool rather than an abstract mathematical subject.

\section{Engineering and Computing Applications of Probability}

The lecture explicitly lists engineering applications to ground probability theory in real-world computational systems.

\subsection*{Applications Listed}

\begin{itemize}
    \item Speech Recognition
    \item System Radar Systems
    \item Communication Networks
\end{itemize}

\subsection*{Dependency Explanation}

\begin{enumerate}
    \item Each listed system processes uncertain or noisy inputs.
    \item Deterministic models are insufficient due to variability in signals and environments.
    \item Probability theory enables:
    \begin{itemize}
        \item Modeling randomness,
        \item Estimating unknown parameters,
        \item Making statistically justified decisions.
    \end{itemize}
\end{enumerate}

These applications are presented solely as motivating contexts and not as worked examples or case studies.

\section{Conceptual Flow of the Lecture}

The lecture follows a conceptual progression rather than immediate mathematical formalism.

\subsection*{Flow Structure}

\begin{enumerate}
    \item \textbf{Course Framing:} Establishes where probability theory fits within the overall course objectives.
    \item \textbf{Motivational Reasoning:} Explains why probability is required in computing and engineering contexts.
    \item \textbf{Application Grounding:} Lists domains where probability theory is essential.
\end{enumerate}

This ordering is deliberate:
\begin{itemize}
    \item Motivation precedes formalism.
    \item Applications precede abstraction.
    \item Conceptual understanding precedes mathematical rigor.
\end{itemize}

\section{Academic and Instructional Structure}

The lecture is delivered within an active learning framework supported by:
\begin{itemize}
    \item In-class discussions,
    \item Online participation via Campuswire,
    \item Continuous feedback mechanisms.
\end{itemize}

These elements define the instructional environment in which probability theory will be learned and assessed.

\section{Scope and Limitations of Lecture 3}

Based strictly on the provided material:
\begin{itemize}
    \item No formal definitions (e.g., probability space, random variable) are introduced.
    \item No axioms, theorems, or proofs are presented.
    \item No mathematical derivations or worked examples appear.
\end{itemize}

This lecture functions solely as a conceptual and motivational introduction.

\section{Logical Dependency Summary}

\begin{itemize}
    \item Probability theory is required because computing systems operate under uncertainty.
    \item Engineering systems exemplify this uncertainty.
    \item Understanding the necessity of probability precedes learning its formal mathematical structure.
    \item Lecture 3 establishes this necessity and prepares students for formal probability theory in subsequent lectures.
\end{itemize}

\vspace{1em}
\hrule
\vspace{0.5em}

\begin{center}
\textit{End of Lecture 3 Scribe}
\end{center}

\end{document}
