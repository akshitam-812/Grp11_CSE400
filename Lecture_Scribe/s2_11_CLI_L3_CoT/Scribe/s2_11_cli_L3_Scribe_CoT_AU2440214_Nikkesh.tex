\documentclass{article}

\usepackage{hyperref}

\begin{document}

\section*{CSE400 – Fundamentals of Probability in Computing}

\section*{Lecture 3: Introduction to Probability Theory}

\textbf{Instructor:} Dr. Dhaval Patel, PhD\\
\textbf{Course:} CSE400 – Fundamentals of Probability in Computing\\
\textbf{Institution:} SEAS, Ahmedabad University, Ahmedabad, Gujarat, India\\
\textbf{Date:} January 13, 2026

\hrule

\section*{1. Lecture Context and Purpose}

This lecture is part of the CSE400 course titled \textit{Fundamentals of Probability in Computing}. The lecture formally introduces the course structure, motivation, and administrative framework before proceeding to probability theory in subsequent lectures. The logical flow strictly follows the lecture slides.

\section*{2. Course Identification}

\begin{itemize}
\item Course Code: CSE400
\item Course Title: Fundamentals of Probability in Computing
\item Lecture Number: Lecture 3
\item Lecture Title: Introduction to Probability Theory
\end{itemize}

\section*{3. Instructor Information}

\begin{itemize}
\item Name: Dr. Dhaval Patel, PhD
\item Role: Instructor
\item Office: Faculty Office, Room 210
\item Faculty Profile: SEAS, Ahmedabad University
\item Email: \href{mailto:dhaval.patel@ahduni.edu.in}{dhaval.patel@ahduni.edu.in}
\item Areas of Interest:
\begin{itemize}
\item xG Networks
\item Applied ML / DL / RL / AutoML
\item Intelligent Transportation Systems
\item Life Sciences
\item Behaviour Modelling using AI
\end{itemize}
\end{itemize}

\section*{4. Course Team Structure}

The course is supported by a team consisting of senior and junior undergraduate teaching assistants.

\subsection*{4.1 Teaching Assistants}

\begin{itemize}

\item Deep Patel
\begin{itemize}
\item BTech CSE (3rd Year)
\item Research Area: Reinforcement Learning and Pinching Antenna Systems
\item Email: \href{mailto:deep.p4@ahduni.edu.in}{deep.p4@ahduni.edu.in}
\end{itemize}

\item Prapti Patel
\begin{itemize}
\item BTech CSE (4th Year)
\item Research Areas:
\begin{itemize}
\item Smart sensing frameworks using Kolmogorov–Arnold Networks for 5G
\item Fourier analysis-based sensing for 5G and beyond
\end{itemize}
\item Email: \href{mailto:prapti.p@ahduni.edu.in}{prapti.p@ahduni.edu.in}
\end{itemize}

\item Raj Koticha
\begin{itemize}
\item BTech CSE (4th Year)
\item Research Area: Multi-agent reinforcement learning for resource management in NR-V2X platooning
\item Email: \href{mailto:raj.k1@ahduni.edu.in}{raj.k1@ahduni.edu.in}
\end{itemize}

\item Ritu Patel
\begin{itemize}
\item BTech CSE (4th Year)
\item Research Area: Intelligent Transportation Systems
\item Email: \href{mailto:rituben.p@ahduni.edu.in}{rituben.p@ahduni.edu.in}
\end{itemize}

\item Rushi Moliya
\begin{itemize}
\item BTech CSE (4th Year)
\item Research Areas:
\begin{itemize}
\item UAV deployment optimization
\item Multi-UAV energy-efficient sensing
\end{itemize}
\item Email: \href{mailto:rushi.m@ahduni.edu.in}{rushi.m@ahduni.edu.in}
\end{itemize}

\item Ura Modi
\begin{itemize}
\item BTech CSE (3rd Year)
\item Research Area: Pinching Antenna Systems
\item Email: \href{mailto:ura.m@ahduni.edu.in}{ura.m@ahduni.edu.in}
\end{itemize}

\end{itemize}

\section*{5. Learning Philosophy}

\subsection*{5.1 Growth Mindset}

The lecture emphasizes the importance of adopting a growth mindset, characterized by:

\begin{itemize}
\item Viewing failure as an opportunity to grow
\item Willingness to try new things
\item Belief that abilities can be developed through effort
\item Treating feedback as constructive
\end{itemize}

\subsection*{5.2 Fixed Mindset (Contrast)}

In contrast, a fixed mindset includes:

\begin{itemize}
\item Viewing failure as a limitation of ability
\item Avoidance of challenges
\item Belief that abilities are static
\end{itemize}

\section*{6. Course Motivation}

\subsection*{6.1 Why Study CSE400?}

The motivation for learning probability is introduced using:

\begin{itemize}
\item Daily life conversations as intuitive examples of probabilistic reasoning
\end{itemize}

\subsection*{6.2 Engineering Applications}

Probability theory is motivated through applications in:

\begin{itemize}
\item Speech Recognition
\item Radar Systems
\item Communication Networks
\end{itemize}

\section*{7. Active Learning Platform}

\subsection*{7.1 Campuswire}

Campuswire is used as the primary active learning and communication platform.

\textbf{Purposes:}
\begin{itemize}
\item Anonymous participation
\item Back-channel communication during lectures
\item Collaborative problem solving
\item Real-time feedback through polling
\item Direct messaging with instructor and TAs
\end{itemize}

\textbf{Course Sections:}
\begin{itemize}
\item Section 1: Campuswire link provided in slides
\item Section 2: Campuswire link provided in slides
\end{itemize}

\section*{8. Lecture Schedule}

\subsection*{8.1 Lecture Sessions}

\textbf{Section 1:}
\begin{itemize}
\item Time: 9:30 AM – 11:00 AM
\item Days: Tuesday, Thursday
\item Venue: GICT Room 136
\end{itemize}

\textbf{Section 2:}
\begin{itemize}
\item Time: 1:00 PM – 2:30 PM
\item Days: Tuesday, Thursday
\item Venue: GICT Room 137
\end{itemize}

\subsection*{8.2 TA Hours}

\begin{itemize}
\item Mode: In-person / Online
\item Timings: To be announced
\end{itemize}

\section*{9. Instructor Interaction Guidelines}

\begin{itemize}
\item Queries should be posted on Campuswire
\item Contact hours available 24$\times$7 through Campuswire
\item Direct messages allowed for private discussions
\item External engagement opportunities include UGRP-8 (2026)
\end{itemize}

\section*{10. Assessment Overview}

\subsection*{10.1 Project Component}

\begin{itemize}
\item Weightage: 30\%
\end{itemize}

\subsection*{10.2 Project Milestones}

\begin{itemize}
\item M1: Team formation and problem formulation
\item M2: Mathematical modeling
\item M3: Coding and simulation
\item M4: Inference and randomized algorithms
\item M5: Algorithm application and comparison
\item M6: Bounds derivation and final analysis
\end{itemize}

\section*{11. Scribe Requirement}

Lecture scribes are a formal course requirement.

\begin{itemize}
\item Two groups per lecture will prepare a scribe
\item Minimum length: 8–10 pages
\item Content must strictly reflect lecture material
\end{itemize}

\section*{12. End of Lecture}

The lecture concludes with an open Q\&A session.

\textbf{Contact:} \href{mailto:dhaval.patel@ahduni.edu.in}{dhaval.patel@ahduni.edu.in}

\end{document}